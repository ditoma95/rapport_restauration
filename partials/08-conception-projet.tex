\chapter{Phase de réalisation du projet}
\clearpage

\section{Introduction}
Dans le cycle de développement d’un projet, la conception joue un rôle fondamental. Elle sert à établir l’architecture globale, à définir les principaux composants et à organiser les interactions entre les modules du système. Cette étape implique aussi l’identification des acteurs du système, qu’il s’agisse des utilisateurs ou des entités avec lesquelles il interagit, afin d’adapter les fonctionnalités aux besoins concrets. Une conception bien pensée assure une organisation efficace du projet, tout en simplifiant sa mise en œuvre et sa maintenance.

\section{Objectifs}
Les objectifs de la phase de conception sont les suivants :
\begin{itemize}
    \item Identifier et structurer les différents composants du système.
    \item Définir l'architecture logicielle en fonction des besoins du projet.
    \item Spécifier les interactions entre les modules pour assurer une cohérence
          globale.
    \item Concevoir les diagrammes UML, y compris les diagrammes de classes et de
          processus, afin de modéliser clairement la structure et le fonctionnement du
          système.
    \item Optimiser la conception en tenant compte des performances et de l'évolutivité.
    \item Assurer la conformité aux standards et aux bonnes pratiques du développement
          logiciel.
\end{itemize}

\section{Modélisation et diagrammes}

\subsection{Dictionnaire de données}
Le dictionnaire de données décrit en détail les attributs des entités de la
base de données, y compris leurs types, leurs rôles et leurs relations. Il
permet d'assurer la cohérence et la structuration correcte des informations
stockées.

\renewcommand{\arraystretch}{1.3} % Améliore l'espacement du tableau

\begin{longtable}{|p{3.5cm}|p{3.5cm}|p{3cm}|p{5cm}|}
    \hline
    \textbf{Nom de la table} & \textbf{Nom de l'attribut} & \textbf{Type de données}    & \textbf{Description} \\
    \hline
    \endfirsthead

    \hline
    \textbf{Nom de la table} & \textbf{Nom de l'attribut} & \textbf{Type de données}    & \textbf{Description} \\
    \hline
    \endhead

    % User
    \hline
    User                     & username                   & string                      & Nom d'utilisateur    \\
    \hline
    User                     & password                   & string                      & Mot de passe         \\
    \hline
    User                     & active                     & string                      & Statut actif/inactif \\
    \hline

    % Role
    Role                     & name                       & string                      & Nom du rôle          \\ \hline

    % Permission
    Permission               & name                       & string                      & Nom de la permission \\ \hline

    % Employee
    Employee                 & first\_name                & string                      & Prénom de l'employé  \\ \hline Employee &
    last\_name               & string                     & Nom de famille de l'employé                        \\ \hline Employee &
    phone\_number            & string                     & Numéro de téléphone                                \\ \hline Employee & email                             &
    string                   & Adresse email                                                                   \\ \hline Employee & address1 & string & Adresse
    principale                                                                                                 \\ \hline Employee & address2 & string & Adresse secondaire \\
    \hline Employee          & city                       & string                      & Ville                \\ \hline Employee & country & string &
    Pays                                                                                                       \\ \hline Employee & state & string & Région ou état \\ \hline Employee &
    bank\_name               & string                     & Nom de la banque                                   \\ \hline Employee & rib & string &
    Relevé d'identité bancaire                                                                                 \\ \hline

    % Department
    Department               & name                       & string                      & Nom du département   \\ \hline

    % File
    File                     & name                       & string                      & Nom du fichier       \\ \hline File & url & string & URL du
    fichier                                                                                                    \\ \hline File & mime\_type & string & Type MIME \\ \hline File & path
                             & string                     & Chemin d'accès                                     \\ \hline File & upload\_date & string & Date
    d'upload                                                                                                   \\ \hline

    % Duty
    Duty                     & duration                   & string                      & Durée de la mission  \\ \hline Duty & begin\_date                       &
    string                   & Date de début                                                                   \\ \hline Duty & type & string & Type de mission \\
    \hline Duty              & etat                       & string                      & État de la mission   \\ \hline

    % Job
    Job                      & title                      & string                      & Intitulé du poste    \\ \hline

    % Absence
    Absence                  & day\_requested             & string                      & Jour demandé         \\ \hline Absence &
    start\_date              & string                     & Date de début                                      \\ \hline Absence & end\_date & string &
    Date de fin                                                                                                \\ \hline Absence & address & string & Adresse \\ \hline Absence        &
    date\_of\_application    & string                     & Date de la demande                                 \\ \hline Absence & status
                             & string                     & Statut de l'absence                                \\ \hline Absence & date\_of\_approval                    & string
                             & Date d'approbation                                                              \\ \hline Absence & type\_of\_absence & string & Type
    d'absence                                                                                                  \\ \hline Absence & reasons & string & Raisons de l'absence \\ \hline
    Absence                  & proof                      & string                      & Justificatif         \\ \hline Absence & comment & string &
    Commentaire                                                                                                \\ \hline

\end{longtable}
\begin{center}  
    \captionof{table}{Tableau du dictionnaire des données} % Ajoute la légende à la liste des tableaux  
    \label{tab:table_dictionnaire_data}  
\end{center}  
\subsection{Diagramme de classes}
Le diagramme de classes permet de représenter les différentes entités du
système et leurs relations. Il offre une vision claire de la structure et
facilite la compréhension des interactions entre les objets.

\begin{figure}[H]
    \centering
    \includegraphics[width=0.8\textwidth]{images/diagrammes/class/diag.png}
    \caption{Diagramme de classe - Gestion du projet OptiHR}
    \label{fig:class_diagramm_optiRH}
\end{figure}

\section{Description des entités et relations}

Le diagramme de classes représente un système de gestion des employés et de
leurs activités professionnelles. Voici une description des entités principales
et de leurs relations :

\subsection{Entités principales}

\subsubsection{User (Utilisateur)}
\textbf{Attributs} :
\begin{itemize}
    \item name
    \item surname
    \item password
    \item email
    \item phone
\end{itemize}
\textbf{Relations} : Un utilisateur est associé à un employé.

\subsubsection{Employee et DG}
\textbf{Attributs} :
\begin{itemize}
    \item L'employé héritent de User
\end{itemize}
\textbf{Relations} :


\subsubsection{Role (Rôle)}
\textbf{Attributs} :
\begin{itemize}
    \item name
\end{itemize}
\textbf{Relations} : Un utilisateur peut avoir plusieurs rôles.

\subsubsection{Permission (Permission)}
\textbf{Attributs} :
\begin{itemize}
    \item name
\end{itemize}
\textbf{Relations} : Un rôle peut avoir plusieurs permissions.


\section{Technologies et outils}
Les outils et technologies suivants ont été utilisés pour la conception de cette application faisant office de marché en ligne.
Chaque outil est accompagné d'une image et d'une
description détaillée.


\section{Conclusion}
La phase de conception pose les bases essentielles du projet en structurant
l'architecture et en précisant les technologies et les modèles de données
adoptés. Une conception rigoureuse garantit un développement fluide et
efficace, tout en assurant la maintenance et l'évolutivité du système sur le
long terme.

\clearpage
