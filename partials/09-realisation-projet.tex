\chapter{Réalisation}
\clearpage
\section{Introduction}
Dans cette section, nous présentons le processus de développement et d’implémentation du système basé sur le modèle conceptuel défini précédemment. Nous détaillerons les choix technologiques adoptés, l’architecture logicielle mise en place, ainsi que les différentes étapes de développement, allant de la conception de la base de données à l’intégration des fonctionnalités essentielles.

L’implémentation suivra une approche modulaire afin de garantir la maintenabilité et l’évolutivité du système. Nous mettrons également en avant les bonnes pratiques de développement, telles que la structuration du code, la gestion des dépendances et l’optimisation des performances.

Enfin, nous testerons et validerons le bon fonctionnement du système à travers des scénarios réels d’utilisation, en nous assurant qu’il répond aux exigences fonctionnelles et non fonctionnelles définies.
%===========================================================================================================

\section{Architecture du projet}
\subsection{Présentation de l'architecture globale}
% Décrire l'architecture générale (monolithe, microservices, MVC, etc.)
%===========================================================================================================

\subsection{Choix des technologies et outils}
% Expliquer les choix technologiques (Node.js, Vue.js, React.js, PostgreSQL, etc.)
% Justifier ces choix en fonction des besoins du projet

Dans cette section, nous présentons les principales technologies utilisées pour le développement du projet. Chaque technologie a été sélectionnée pour ses performances, sa robustesse et sa facilité d’intégration dans notre stack.

\vspace{1cm} % Ajoute un espace vertical


\begin{longtable}{|m{4cm}|m{10cm}|}
    \hline
    \textbf{Technologie} & \textbf{Description} \\
    \hline
    \endfirsthead

    \hline
    \textbf{Technologie} & \textbf{Description} \\
    \hline
    \endhead

    \hline
    \endfoot

    \hline
    \includegraphics[width=3cm]{images/logo/laravel.png} & 
    \textbf{Laravel - Framework Backend} : Laravel est un framework PHP moderne basé sur l’architecture MVC (Modèle-Vue-Contrôleur). Il offre de nombreuses fonctionnalités telles que :  
    \begin{itemize}
        \item Un ORM puissant (Eloquent) pour la gestion de la base de données.
        \item Un système de migration et de seeders pour faciliter le développement.
        \item Un mécanisme de routage avancé et un middleware intégré pour la gestion des requêtes HTTP.
    \end{itemize}
    Grâce à Laravel, nous avons pu structurer le projet de manière efficace et évolutive. \\
    \hline

    \includegraphics[width=3cm]{images/logo/blade.png} & 
    \textbf{Blade - Moteur de Templating} : Blade est le moteur de templates natif de Laravel. Il permet de créer des vues dynamiques avec une syntaxe claire et fluide. Ses principales caractéristiques sont :
    \begin{itemize}
        \item Une syntaxe simplifiée pour l'affichage des données et les structures de contrôle (`@if`, `@foreach`, etc.).
        \item La possibilité d’étendre des layouts grâce à l’héritage de templates.
        \item Une mise en cache automatique pour améliorer les performances.
    \end{itemize}\\
    \hline

    \includegraphics[width=3cm]{images/logo/postgresql.png} & 
    \textbf{PostgreSQL - Base de Données Relationnelle} : PostgreSQL est un système de gestion de base de données relationnelle open-source reconnu pour sa stabilité et ses performances. Il a été choisi pour :  
    \begin{itemize}
        \item Son support avancé des types de données et des transactions ACID.
        \item Sa capacité à gérer de gros volumes de données efficacement.
        \item Son intégration facile avec Laravel via l'ORM Eloquent.
    \end{itemize}\\
    \hline

    \includegraphics[width=3cm]{images/logo/bootstrap.png} & 
    \textbf{Bootstrap - Framework CSS} : Bootstrap est un framework CSS populaire utilisé pour concevoir une interface utilisateur réactive et attrayante. Il nous a permis de :  
    \begin{itemize}
        \item Utiliser un système de grille pour une mise en page responsive.
        \item Accélérer le développement avec des composants préconçus (modals, boutons, alertes, etc.).
        \item Assurer une compatibilité avec tous les navigateurs modernes.
    \end{itemize}\\
    \hline

    \includegraphics[width=3cm]{images/logo/javascript.png} & 
    \textbf{JavaScript - Langage de Programmation Frontend} : JavaScript est un langage de programmation utilisé pour ajouter des interactions dynamiques à l'interface utilisateur. Il a été employé pour :  
    \begin{itemize}
        \item Gérer les interactions utilisateur (événements, animations).
        \item Améliorer l'expérience avec des requêtes asynchrones (AJAX, Fetch API).
        \item Dynamiser le rendu des composants sans recharger la page.
    \end{itemize}\\
    \hline

    \includegraphics[width=3cm]{images/logo/git.png}  
    \includegraphics[width=3cm]{images/logo/github.png} & 
    \textbf{Git et GitHub - Outils de Gestion de Version} : Pour la gestion du code source, nous avons utilisé **Git** et **GitHub** :  
    \begin{itemize}
        \item **Git** permet de suivre l'évolution du projet grâce à un système de versionnement performant.
        \item **GitHub** facilite la collaboration et l’hébergement du code en ligne, avec des fonctionnalités comme les pull requests et les issues.
    \end{itemize}
    Ces outils nous ont permis de travailler efficacement en équipe et d’assurer la stabilité du code tout au long du développement.\\
    \hline

\end{longtable}
\begin{center}  
    \captionof{table}{Tableau des technologies utilisées pour la réalisation} % Ajoute la légende à la liste des tableaux  
    \label{tab:table_techs_realisation} % Permet de faire référence à ce tableau plus tard
\end{center}  
L’utilisation de ces technologies nous a permis de construire une application performante, maintenable et évolutive. Le choix de Laravel avec Blade pour le backend, PostgreSQL pour la base de données et Bootstrap avec JavaScript pour le frontend a facilité l’implémentation et l’optimisation du projet. Enfin, l’utilisation de Git et GitHub a renforcé la gestion du code et le travail collaboratif.

%===========================================================================================================
\section{Développement et implémentation}

\subsection{Mise en place de l’environnement de développement}
% Décrire la configuration initiale du projet (installation des dépendances, structuration du code)


Pour assurer un développement fluide et efficace, un environnement de travail stable et bien configuré a été mis en place sous **Ubuntu 24**. Cette section détaille les étapes suivies pour l’installation et la configuration des outils nécessaires.

\subsubsection{Prérequis}
Les outils suivants ont été utilisés :
\begin{itemize}
    \item Système d’exploitation : \textbf{Ubuntu 24}
    \item Éditeur de code : \textbf{Visual Studio Code}
    \item Serveur Web et PHP : \textbf{Apache, PHP 8.2}
    \item Base de données : \textbf{PostgreSQL}
    \item Gestionnaire de paquets : \textbf{Composer (pour PHP) et npm (pour JavaScript)}
    \item Système de versionnement : \textbf{Git et GitHub}
\end{itemize}

\subsubsection{Installation des outils}

\subsubsection*{Installation de PHP 8.2 et Apache}
Ubuntu 24 ne propose pas PHP 8.2 par défaut. Pour l’installer avec Apache :
\begin{tcolorbox}[colback=black, coltext=white, title=Installation de PHP 8.2 et Apache, fonttitle=\bfseries]
\texttt{sudo apt update \&\& sudo apt upgrade -y} \\
\texttt{sudo apt install apache2 php8.2 libapache2-mod-php8.2 php8.2-cli php8.2-mbstring php8.2-xml php8.2-curl php8.2-pgsql unzip -y}
\end{tcolorbox}

Une fois l’installation terminée, redémarrer Apache :

\begin{tcolorbox}[colback=black, coltext=white, title=Redémarrage du serveur Apache, fonttitle=\bfseries]
\texttt{sudo systemctl restart apache2}
\end{tcolorbox}

Vérifier que PHP est bien installé :

\begin{tcolorbox}[colback=black, coltext=white, title=Vérification de la version de PHP, fonttitle=\bfseries]
\texttt{php -v}
\end{tcolorbox}

\subsubsection*{Installation de Composer}
Composer est indispensable pour gérer les dépendances PHP :

\begin{tcolorbox}[colback=black, coltext=white, title=Installation de Composer, fonttitle=\bfseries]
\texttt{curl -sS https://getcomposer.org/installer | php} \\
\texttt{sudo mv composer.phar /usr/local/bin/composer} \\
\texttt{composer -V}
\end{tcolorbox}

\subsubsection{Installation et configuration de PostgreSQL}
PostgreSQL est utilisé comme base de données :

\begin{tcolorbox}[colback=black, coltext=white, title=Installation de PostgreSQL, fonttitle=\bfseries]
\texttt{sudo apt install postgresql postgresql-contrib -y}
\end{tcolorbox}

Démarrer PostgreSQL et vérifier son installation :

\begin{tcolorbox}[colback=black, coltext=white, title=Démarrage et vérification, fonttitle=\bfseries]
\texttt{sudo systemctl start postgresql} \\
\texttt{sudo systemctl enable postgresql} \\
\texttt{psql --version}
\end{tcolorbox}

\subsubsection*{Installation de Laravel}
Créer un projet Laravel :

\begin{tcolorbox}[colback=black, coltext=white, title=Création d’un projet Laravel, fonttitle=\bfseries]
\texttt{composer create-project --prefer-dist laravel/laravel:\^10.0 nom\_du\_projet}
\end{tcolorbox}

Générer la clé d’application :

\begin{tcolorbox}[colback=black, coltext=white, title=Génération de la clé d’application, fonttitle=\bfseries]
\texttt{php artisan key:generate}
\end{tcolorbox}

\subsubsection{Configuration de Git et clonage du projet}
Vérifier si Git est installé :

\begin{tcolorbox}[colback=black, coltext=white, title=Vérification de Git, fonttitle=\bfseries]
\texttt{git --version}
\end{tcolorbox}

Si Git n’est pas installé, l’ajouter avec :

\begin{tcolorbox}[colback=black, coltext=white, title=Installation de Git, fonttitle=\bfseries]
\texttt{sudo apt install git -y}
\end{tcolorbox}

Cloner le projet depuis GitHub :

\begin{tcolorbox}[colback=black, coltext=white, title=Clonage du projet, fonttitle=\bfseries]
\texttt{git clone https://github.com/utilisateur/nom\_du\_projet.git} \\
\texttt{cd nom\_du\_projet}
\end{tcolorbox}

\subsubsection{Lancement du serveur de développement}
Démarrer Laravel en mode développement :

\begin{tcolorbox}[colback=black, coltext=white, title=Lancement du serveur Laravel, fonttitle=\bfseries]
\texttt{php artisan serve}
\end{tcolorbox}

L’application est maintenant accessible à l’adresse :

\begin{tcolorbox}[colback=black, coltext=white, title=URL d’accès, fonttitle=\bfseries]
\texttt{http://127.0.0.1:8000/}
\end{tcolorbox}






\subsection{Développement des fonctionnalités principales}
% Présenter les fonctionnalités principales du projet et leur implémentation

\subsubsection{Feature 1 : [Nom de la fonctionnalité]}
% Détailler le développement d’une fonctionnalité clé

\subsubsection{Feature 2 : [Nom de la fonctionnalité]}
% Détailler une autre fonctionnalité clé

\section{Problèmes rencontrés et solutions adoptées}
\subsection{Difficultés techniques}
% Problèmes liés aux technologies utilisées, performances, compatibilité

\subsection{Optimisation et amélioration du code}
% Améliorations mises en place pour optimiser le projet

\section{Tests et validation}
\subsection{Stratégie de test}
% Présentation des tests utilisés (unitaires, intégration, end-to-end)

\subsection{Résultats des tests}
% Analyse des résultats des tests et validation du bon fonctionnement

\section{Conclusion}
% Résumer les étapes clés de la réalisation et l’impact des choix technologiques
\clearpage